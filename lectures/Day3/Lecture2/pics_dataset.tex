% providelength from https://tex.stackexchange.com/questions/184991/a-reusable-providelength-no-stop-break-if-length-already-defined


%\newcommand{\defdatasetstyles}{
\NewDocumentCommand{\defdatasetstyles}{ O{1.0} }{
\def\scale{#1}
% Define the rounding radius
%\pgfmath{\rrad}{\scale*3mm}
\providelength{\rradnull}
\setlength{\rradnull}{1.5mm}
\providelength{\rrad}
\setlength{\rrad}{\scale \rradnull }

% Styles for gradients and so on
\tikzstyle{gradient} = [rectangle, rounded corners, minimum width=0.9cm, minimum height=0.9cm,text centered, draw=none, fill=fzjlightblue]
\tikzstyle{avgradient} = [rectangle, rounded corners, minimum width=0.9cm, minimum height=0.9cm,text centered, draw=none, fill=fzjviolet]
\tikzstyle{weight} = [rectangle, rounded corners, minimum width=0.9cm, minimum height=0.9cm,text centered, draw=none, fill=fzjorange]

\providelength{\layerdist}\setlength{\layerdist}{24mm}
\providelength{\distleft}\setlength{\distleft}{12mm}
\providelength{\boxdist}\setlength{\boxdist}{7mm}

\tikzstyle{item} = [rectangle, rounded corners=\rrad, minimum width=0.6cm, minimum height=0.5cm,text centered, draw=black, fill=white]
}


%%%%%%%%%%%%%%%%% Shard %%%%%%%%%%%%%%%%%%%%%
\NewDocumentCommand{\picdatasetshard}{ O{1.0} O{6} O{3}}{

\def\npernode{#2}
\def\nshards{#3}

\defdatasetstyles

\begin{tikzpicture}[scale=#1, every node/.style={scale=#1}]

\pgfmathtruncatemacro{\nshardsmone}{\nshards-1}
\pgfmathtruncatemacro{\npernodemone}{\npernode-1}

\foreach \i in {0,...,\nshardsmone} {

    \pgfmathtruncatemacro{\npone}{\npernode+1}

    \foreach \j in {0,...,\npernodemone} {

        \pgfmathtruncatemacro{\k}{\npernode*\i+\j}
        
        \node [item]  (input-\k) at (\boxdist * \k + \distleft , 0) {\k}; 
    
        \pgfmathtruncatemacro{\imone}{\i-1}

    }
}
\foreach \i in {0,...,\nshardsmone} {


    \pgfmathtruncatemacro{\knull}{\npernode*\i}
    \pgfmathtruncatemacro{\kone}{\npernode*\i}
    %\node [yshift=8mm, xshift=3mm] (node) at (input-\knull) {Shard \i};
    
    \pgfmathtruncatemacro{\nmone}{\npernode-1}
    \foreach \j in {0,...,\nmone} {
    
    \pgfmathtruncatemacro{\k}{\npernode*\i+\j}
    \pgfmathtruncatemacro{\l}{\nshards*\j+\i}
    
    \node [item]  (output-\k) at (\boxdist * \k + \distleft , -\layerdist) {\l}; 
    \ifnum \l<5
       \draw [->] (input-\l.south) to [out=-90,in=90] (output-\k.north);
    \fi
    }
}

    
\foreach \i in {0,...,\nshardsmone} {
    \pgfmathtruncatemacro{\knull}{\npernode*\i}

    \node [yshift=-\layerdist-8mm, xshift=3mm] (node) at (input-\knull) {Shard \i};
}
\node [anchor=west] (inp) at (-0.5,0) {Input};
\node [anchor=west] (buf) at (-0.5,-\layerdist) {Shard};

 \foreach \i in {1,...,\nshardsmone} {
     \pgfmathtruncatemacro{\kone}{\npernode*(\i)-1}
     \pgfmathtruncatemacro{\ktwo}{\npernode*(\i)}
     \draw ( $(output-\kone.west)!0.5!(output-\ktwo.east)+(0,0.5)$ ) -- +(0,-2);
     
}

\end{tikzpicture}

} %%%%%%%%%%%%%%%%% Shard %%%%%%%%%%%%%%%%%%%%%

%%%%%%%%%%%%%%%%% Split %%%%%%%%%%%%%%%%%%%%%
\NewDocumentCommand{\picdatasetsplit}{ O{1.0} O{6} O{3}}{

\def\npernode{#2}
\def\nshards{#3}

\defdatasetstyles

\begin{tikzpicture}[scale=#1, every node/.style={scale=#1}]

\pgfmathtruncatemacro{\nshardsmone}{\nshards-1}
\pgfmathtruncatemacro{\npernodemone}{\npernode-1}

\foreach \i in {0,...,\nshardsmone} {

    \pgfmathtruncatemacro{\npone}{\npernode+1}

    \foreach \j in {0,...,\npernodemone} {

        \pgfmathtruncatemacro{\k}{\npernode*\i+\j}
        
        \node [item]  (input-\k) at (\boxdist * \k + \distleft , 0) {\k}; 
    
        \pgfmathtruncatemacro{\imone}{\i-1}

    }
}
\foreach \i in {0,...,\nshardsmone} {


    \pgfmathtruncatemacro{\knull}{\npernode*\i}
    \pgfmathtruncatemacro{\kone}{\npernode*\i}
    %\node [yshift=8mm, xshift=3mm] (node) at (input-\knull) {Shard \i};
    
    \pgfmathtruncatemacro{\nmone}{\npernode-1}
    \foreach \j in {0,...,\nmone} {
    
    \pgfmathtruncatemacro{\k}{\npernode*\i+\j}
    \pgfmathtruncatemacro{\l}{\nshards*\j+\i}
    
    \node [item]  (output-\k) at (\boxdist * \k + \distleft , -\layerdist) {\k}; 
    
       \draw [->] (input-\k.south) to [out=-90,in=90] (output-\k.north);
    
    }
}

    
\foreach \i in {0,...,\nshardsmone} {
    \pgfmathtruncatemacro{\knull}{\npernode*\i}

    \node [yshift=-\layerdist-8mm, xshift=3mm] (node) at (input-\knull) {Split \i};
}
\node [anchor=west] (inp) at (-0.5,0) {Input};
\node [anchor=west] (buf) at (-0.5,-\layerdist) {Split};

 \foreach \i in {1,...,\nshardsmone} {
     \pgfmathtruncatemacro{\kone}{\npernode*(\i)-1}
     \pgfmathtruncatemacro{\ktwo}{\npernode*(\i)}
     \draw ( $(output-\kone.west)!0.5!(output-\ktwo.east)+(0,0.5)$ ) -- +(0,-2);
     
}

\end{tikzpicture}

}
%%%%%%%%%%%%%%%%% /Split %%%%%%%%%%%%%%%%%%%%%

%%%%%%%%%%%%%%%%% Batch %%%%%%%%%%%%%%%%%%%%%
\NewDocumentCommand{\picdatasetbatch}{ O{1.0} O{4} O{4}}{

\def\npernode{#2}
\def\nshards{#3}

\defdatasetstyles[#1]
\providelength{\layerdist}\setlength{\layerdist}{12mm}

\begin{tikzpicture}[scale=#1,  transform shape]

\pgfmathtruncatemacro{\nshardsmone}{\nshards-1}
\pgfmathtruncatemacro{\npernodemone}{\npernode-1}

\foreach \i in {0,...,\nshardsmone} {

    \pgfmathtruncatemacro{\npone}{\npernode+1}

    \foreach \j in {0,...,\npernodemone} {

        \pgfmathtruncatemacro{\k}{\npernode*\i+\j}
        
        \node [item]  (input-\k) at (\boxdist * \k + \distleft , 0) {\k}; 
    
        \pgfmathtruncatemacro{\imone}{\i-1}

    }
}



\foreach \i in {0,...,\nshardsmone} {


    \pgfmathtruncatemacro{\knull}{\npernode*\i}
    \pgfmathtruncatemacro{\kone}{\npernode*\i}
    
    %\node [yshift=8mm, xshift=3mm] (node) at (input-\knull) {Shard \i};
    
    \pgfmathtruncatemacro{\nmone}{\npernode-1}
    \foreach \j in {0,...,\nmone} {
    
        \pgfmathtruncatemacro{\k}{\npernode*\i+\j}
        \pgfmathtruncatemacro{\l}{\nshards*\j+\i}
        
        \node [item]  (output-\k) at (\boxdist * \knull + \distleft + 1.1cm , -\layerdist-\boxdist*\j ) {\k  }; 
        
           %\draw [->] (input-\k.south) to [out=-90,in=90] (output-\k.north);
        
    }
    \pgfmathtruncatemacro{\klast}{\knull+\nmone}
    \node [item, fill=none, fit={(output-\knull) (output-\klast) } ] {};
    
}

    
\foreach \i in {0,...,\nshardsmone} {
    \pgfmathtruncatemacro{\knull}{\npernode*\i}

    \node [yshift=-\layerdist-8mm, xshift=3mm] (node) at (input-\knull) {\rotatebox{-90}{Batch \i} };
}
\node [anchor=west] (inp) at (-0.5,0) {Input};
\node [anchor=west] (buf) at (-0.5,-\layerdist) {Batch};

 \foreach \i in {1,...,\nshardsmone} {
     \pgfmathtruncatemacro{\kone}{\npernode*(\i)-1}
     \pgfmathtruncatemacro{\ktwo}{\npernode*(\i)}
     \draw ( $(input-\kone.west)!0.5!(input-\ktwo.east)+(0,0.5)$ ) -- +(0,-4);
     
}

\end{tikzpicture}
}



\NewDocumentCommand{\picprefetch}{ O{1.0}  O{7} O{12}}{

\defdatasetstyles[#1]
\setlength{\layerdist}{12mm}

\newcommand{\ninput}{14}
\newcommand{\nbuffer}{8}


\begin{tikzpicture}[scale=#1,  transform shape]
\tikzstyle{item} = [rectangle, rounded corners, minimum width=0.6cm, minimum height=0.5cm,text centered, draw=black, fill=white]

%\def\bufferlabels{{$(0+\nbuffer)$, $(1+\nbuffer)$, $(2+\nbuffer)$,3,4,5,6,7,8,9,10}} 
%\def\bufferlabels{{{12, 13, 14,3,4,5,6,7,8,9,10}}}

\def \nout {#2}
\def \nin {#3}
\pgfmathtruncatemacro{\tmp}{\nbuffer-1}
\pgfmathtruncatemacro{\nbuf}{\nin >\nbuffer? \nin - \nbuffer : \nbuffer }
%\def \nbuf {\tmp} % todo modulo


\node (inp) at (0,0) {Input};
\node (buf) at (0,-\layerdist) {Buffer};
\node (out) at (0,-2*\layerdist) {Output};
\foreach \i in {0,...,\ninput}
{
\node [item]  (input-\i) at (\boxdist * \i + \distleft , 0) {\i}; 
}
\pgfmathtruncatemacro{\tmp}{\nbuffer-1}
\foreach \i in {0,...,\tmp}
{
    \pgfmathtruncatemacro{\label}{ 
    \i> \nin-\nbuffer ? \i : \i+\nbuffer
    }
    \node [item]  (buffer-\i) at ( \boxdist * \i + \distleft, -1*\layerdist) {
    %\bufferlabels[\i]
    \label
    
    }; 
}

\node[item] (output) at (\distleft, -2*\layerdist) { \nout};
% This works
\draw [->] (input-\nin.south) |- ($(input-\nin)!0.5!(buffer-\nout)$)  -| (buffer-\nbuf.north) ;
\draw [->] (buffer-\nout.south) |- ($(buffer-\nout)!0.5!(output)$)  -| (output.north) ;


%\path (input-9.south) edge [->] (buffer-2.north);
\end{tikzpicture}
}

\NewDocumentCommand{\picdatasetshuffle}{ O{1.0}  O{6} O{12}}{

\defdatasetstyles[#1]
\setlength{\layerdist}{12mm}

\def\ninput{14} %% number of input samples
\def\nbuffer{8} %% buffer lengths


\begin{tikzpicture}[scale=#1,  transform shape]
\tikzstyle{item} = [rectangle, rounded corners, minimum width=0.6cm, minimum height=0.5cm,text centered, draw=black, fill=white]

%\def\bufferlabels{{$(0+\nbuffer)$, $(1+\nbuffer)$, $(2+\nbuffer)$,3,4,5,6,7,8,9,10}} 
\def\bufferlabels{{{11, 7, 12 ,5,0,1,9,7}}}

\def \nout {#2} %% Buffer Element that is pulled out
\def \nin {#3}  %% Input Element that is fetched next
\pgfmathtruncatemacro{\tmp}{\nbuffer-1}
\pgfmathtruncatemacro{\nbuf}{ 2 } %% Where to put text input
%\def \nbuf {\tmp} % todo modulo


\node (inp) at (0,0) {Input};
\node (buf) at (0,-\layerdist) {Buffer};
\node (out) at (0,-2*\layerdist) {Shuffle};
\foreach \i in {0,...,\ninput}
{
\node [item]  (input-\i) at (\boxdist * \i + \distleft , 0) {\i}; 
}
\pgfmathtruncatemacro{\tmp}{\nbuffer-1}
\foreach \i in {0,...,\tmp}
{
    \pgfmathtruncatemacro{\label}{ 
    \bufferlabels[\i] %% only works in math mode
    }
    \node [item]  (buffer-\i) at ( \boxdist * \i + \distleft, -1*\layerdist) {
    \label
    
    }; 
}

\pgfmathtruncatemacro{\label} {
    \bufferlabels[\nout]
}

\node[item] (output) at (\distleft, -2*\layerdist) { \label};
% This works
\draw [->, dashed] (input-\nin.south) |- ($(input-\nin)!0.5!(buffer-\nout)$) node [below] {fill buffer} -| (buffer-\nbuf.north) ;
\draw [->] (buffer-\nout.south) |- ($(buffer-\nout)!0.5!(output)$) node [below] {pick at random} -| (output.north) ;


%\path (input-9.south) edge [->] (buffer-2.north);
\end{tikzpicture}
}